\documentclass{ctexbeamer}
\usetheme[red]{sjtubeamer}
\usesjtutheme{poster}
% \usesjtutheme[landscape]{poster}
% \usesjtutheme[sepinst]{poster}
% \usesjtutheme[noindent]{poster}
\usepackage[style=gb7714-2015]{biblatex}
\addbibresource{\getcontribpath{poster}{ref.bib}}
\begin{document}
  \title{poster 子主题}
  \author{作者}
  \logo{\zhlogo}
  \institute[Test Institute]{测试机构}
  \footline[左脚注]{右脚注}
  \begin{frame}[fragile]
    \begin{columns}[T]
      \begin{column}{.45\textwidth}

        \begin{stampblock}{主题概述}
          \texttt{poster} 子主题基于 \texttt{beamerposter} 宏包\cite{beamerposter},对 SJTUBeamer 进行适配以制作海报,展示您的成果。
          \codeblockinput[firstline=1,lastline=3]{开始使用}{poster.tex}
        \end{stampblock}

        \begin{stampblock}{风格建议}
          建议使用 \texttt{stampblock} 环境分割不同的区块。你也可以使用 \texttt{block} 环境分割不同的区块。相较于普通 \texttt{beamer} 幻灯片文字较少的情形,\texttt{poster} 更有可能会写入大段文字,所以默认情况下这些区块内部会有首行缩进。由于 \texttt{column} 环境内部不受外部首行缩进的影响,所以不推荐之间在 \texttt{column} 环境中书写内容,而应该再套一层 \texttt{stampblock} 或 \texttt{block}。
          
          可以插入图片来丰富你的海报内容,但请注意由于海报的纸张尺寸较大,像一些高 DPI 显示的场景一样,会对文字大小进行缩放,但是度量上却没有发生变化,请注意要使用相对于平时更大的度量插入图片。

          不推荐直接插入 Ti\emph{k}Z 的代码,推荐使用 \texttt{standalone} 文档类产生 PDF 文件后再插入,否则会导致图片内的文字过大而排版错乱。
        \end{stampblock}

      \end{column}
      \begin{column}{.45\textwidth}
        
        \begin{stampblock}{多栏对齐}
          推荐使用 \texttt{columns} 环境分割多栏,环境可选选项可以用于指定顶部对齐 \texttt{t},居中对齐 \texttt{c},底部对齐 \texttt{b}。
          \codeblockinput[firstline=16,lastline=16,firstnumber=1]{多栏}{poster.tex}
        \end{stampblock}
        
        
        \begin{stampblock}{参考文献}
        
          可以使用图标样式的参考文献。

          \begin{bibliolist}{00}
            \onlineitem \textsc{Tantau T}, \textsc{Wright J}, and \textsc{Mileti\'c V}.\newblock
            The beamer class: User Guide for version 3.67[OL].\newblock
            2022. \url{https://github.com/josephwright/beamer}
            \articleitem \textsc{上海交通大学}.\newblock
            上海交通大学视觉形象识别系统[OL].\newblock
            2016. \url{https://vi.sjtu.edu.cn}
            \bookitem \textsc{Knuth DE}.\newblock
            The \TeX{}book[M].\newblock
            Addison-Wesley. 1986.
          \end{bibliolist}

          你也可以使用 \texttt{biblatex} 插入参考文献。

          \printbibliography
          
        \end{stampblock}
        
      \end{column}
    \end{columns}

    % 分割线
    \vspace*{2cm}
    \begin{tikzpicture}
      \draw[cprimary,decoration=stampline,segment length=1cm,decorate] (0,0) -- (0.94\textwidth,0);
    \end{tikzpicture}
    \vspace*{2cm}

    \begin{columns}
      \begin{column}{0.45\textwidth}
        \begin{block}{选项}
          \texttt{poster} 子主题拥有一些选项。
          \begin{description}
            \item[\texttt{sepinst}] 分离徽标与机构于两侧
            \item[\texttt{landscape}] 横向海报
            \item[\texttt{noindent}] 首行不缩进  
          \end{description}
        \end{block}
        \begin{alertblock}{脚注}
          使用 \texttt{\textbackslash{}footline[左脚注]\{右脚注\}} 设定左脚注和右脚注。
        \end{alertblock}
        \begin{exampleblock}{区块}
          仍然可以使用内置的 \texttt{block}, \texttt{alertblock}, \texttt{exampleblock} 插入对应的区块。
        \end{exampleblock}
      \end{column}
      \begin{column}{0.45\textwidth}
        
        \begin{stampblock}[a]{印记区块}
          使用 \texttt{stampblock} 环境插入带有印记图案的区块,编号会自动递增。你也可以通过指定可选参数来设置编号内容。
        \end{stampblock}
        
        \begin{codeblock}[numbers=none]{代码块}
% 应当减少代码块的使用,增加了标题栏的高度。
% 关闭代码行号编号可以在 codeblock 上添加选项 numbers=none
        \end{codeblock}

        \begin{columns}[b]
          \begin{column}{.5\textwidth}
            \begin{stampbox}
              \includegraphics[width=16cm]{sjtuphoto.jpg}
            \end{stampbox}
          \end{column}
          \begin{column}{.5\textwidth}
            $\leftarrow$ 你仍然可以使用 \texttt{stampbox} 环境插入带边框的图片\vspace{1ex}
          \end{column}
        \end{columns}
      \end{column}
    \end{columns}
  \end{frame}
\end{document}
% !TeX encoding = UTF-8
% !TeX root = ../main.tex

\renewcommand{\TeX}{\hologo{TeX}}
\renewcommand{\LaTeX}{\hologo{LaTeX}}
\newcommand{\BibTeX}{\hologo{BibTeX}}
\newcommand{\XeTeX}{\hologo{XeTeX}}
\newcommand{\pdfTeX}{\hologo{pdfTeX}}
\newcommand{\LuaTeX}{\hologo{LuaTeX}}
\renewcommand{\CTeX}{C\TeX}
\newcommand{\MiKTeX}{\hologo{MiKTeX}}
\newcommand{\MacTeX}{Mac\hologo{TeX}}
\newcommand{\beamer}{\textsc{beamer}}
\newcommand{\XeLaTeX}{\hologo{Xe}\kern-.13em\LaTeX{}}
\newcommand{\pdfLaTeX}{pdf\LaTeX{}}
\newcommand{\LuaLaTeX}{Lua\LaTeX{}}
\newcommand\pkg[1]{\texttt{#1}}
\def\cmd#1{\texttt{\color{DarkBlue}\footnotesize $\backslash$#1}}
\def\env#1{\texttt{\color{DarkBlue}\footnotesize #1}}
\def\cmdxmp#1#2#3{\small{\texttt{\color{DarkBlue}$\backslash$#1}\{#2\}\hspace{1em}\\ $\Rightarrow$\hspace{1em} {#3}\par\vskip1em}}
\newcommand\link[1]{\href{#1}{[Link]}}
\newcommand{\SJTUThesis}{\textsc{SJTUThesis}\xspace}
\newcommand{\SJTUThesisVersion}{1.0.0rc7}
\newcommand{\SJTUThesisDate}{2020/7/31}

\section{学术论文排版}
\subsection{\LaTeX{} 排版入门}

\begin{frame}[fragile]
  \frametitle{引擎与格式}
  \begin{itemize}
    \item \textbf{引擎}:\TeX{} 的实现

      \begin{itemize}
        \item \pdfTeX{}:直接生成 PDF,支持 micro-typography
        \item \XeTeX{}:支持 Unicode、OpenType 与复杂文字编排(CTL)
        \item \LuaTeX{}:支持 Unicode,内联 Lua,支持 OpenType
        \item (u)p\TeX{}:日本方面推动,生成 |.dvi|,(支持 Unicode)
        \item Ap\TeX{}:底层 CJK 支持,内联 Ruby,Color Emoji
      \end{itemize}

    \item \textbf{格式}:\TeX{} 的语言扩展(命令封装)

      \begin{itemize}
        \item plain \TeX{}:Knuth 同志专用
        \item \LaTeX{}:排版科技类文章的事实标准
        \item Con\TeX t:基于 \LuaTeX{} 实现,优雅、易用(吗?)
      \end{itemize}

    \item \textbf{程序}:引擎 + dump 后的格式代码

      \begin{itemize}
        \item \alert{英文文章:\pdfLaTeX{}、\XeLaTeX{} 或 \LuaLaTeX{}}
        \item \alert{中文文章:\XeLaTeX{} 或 \LuaLaTeX{}}
      \end{itemize}
  \end{itemize}
\end{frame}

\begin{frame}[fragile]
  \frametitle{编译}
  \begin{itemize}
    \item 现代 \TeX{} 引擎均可直接生成 PDF
    \item 命令行

      \begin{itemize}
        \item |pdflatex|/|xelatex|/|lualatex| + |<文件名>[.tex]|
        \item 多次编译:每次均需要读取并处理中间文件
        \item 推荐 \pkg{latexmk}\footnote{\MiKTeX 用户需要自行安装 perl 解释器}:运行 |latexmk [<选项>] <文件名>| 即可自动完成所有工作
      \end{itemize}

    \item 编辑器

      \begin{itemize}
        \item 按钮的背后仍然是命令
        \item |PATH| 环境变量:确定可执行文件的位置
        \item VS Code + \LaTeX{} Workshop:配置 |tools| 和 |recipes|
      \end{itemize}
  \end{itemize}
\end{frame}


\begin{frame}[fragile]{文件结构}
  \lstset{language=[LaTeX]TeX}
  \begin{lstlisting}[basicstyle=\ttfamily]
\documentclass[a4paper]{ctexart}
% 文档类型,如 ctexart,[]内是选项,如 a4paper
% 这里开始是导言区
\usepackage{graphicx} % 引用宏包
\graphicspath{{fig/}} % 设置图片目录
% 导言区到此为止
\begin{document}
这里开始是正文
\end{document}
  \end{lstlisting}
\end{frame}

\begin{frame}[fragile]{\LaTeX{}“命令”}
  \framesubtitle{\emph{宏} (Macro)、或者\emph{控制序列} (control sequence)}
\begin{itemize}
\item 简单命令
  \begin{itemize}
    \item \verb|\命令|\hspace{2em}
    \verb|{\songti 中国人民解放军}| ~$\Rightarrow$ {\songti 中国人民解放军}
  \item \verb|\命令[可选参数]{必选参数}|\\
\verb|\section[精简标题]{这个题目实在太长了放到目录里面不太好看}|\\
$\Rightarrow$ {\heiti 1.1 \hspace{1em} \songti 这个题目实在太长了放到目录里面不太好看}
  \end{itemize}
\item 环境
  \begin{columns}[c]
  \begin{column}{0.45\textwidth}
    \begin{lstlisting}[basicstyle=\ttfamily]
\begin{equation*}
  a^2-b^2=(a+b)(a-b)
\end{equation*}
\end{lstlisting}
\end{column}\hspace{1em}
  \begin{column}{0.45\textwidth}
$ a^2-b^2=(a+b)(a-b)$
\end{column}
  \end{columns}
\end{itemize}
\end{frame}

\begin{frame}[fragile]{\LaTeX{} 常用命令}
  \begin{exampleblock}{简单命令}
\centering
\footnotesize
  \begin{tabular}{llll}
    \cmd{chapter} & \cmd{section} & \cmd{subsection} & \cmd{paragraph} \\
    章 & 节 & 小节 & 带题头段落 \\\hline
    \cmd{centering} & \cmd{emph} & \cmd{verb} & \cmd{url} \\
   居中对齐         &  强调      & 原样输出   & 超链接 \\\hline
  \cmd{footnote} & \cmd{item} & \cmd{caption} & \cmd{includegraphics} \\
   脚注 & 列表条目 & 标题 & 插入图片 \\\hline
  \cmd{label} & \cmd{cite} & \cmd{ref} \\
  标号 & 引用参考文献 & 引用图表公式等\\\hline
  \end{tabular}
\end{exampleblock}
\end{frame}
\begin{frame}[fragile]{\LaTeX{} 常用环境}
\begin{exampleblock}{环境}
\centering
\footnotesize
\begin{tabular}{lll}
  \env{table} & \env{figure} & \env{equation}\\
  表格 & 图片 & 公式 \\\hline
  \env{itemize} & \env{enumerate} & \env{description}\\
  无编号列表 & 编号列表 & 描述 \\\hline
\end{tabular}
\end{exampleblock}
\end{frame}
%
\begin{frame}{\LaTeX{}命令举例}
\cmdxmp{chapter}{前言}{\heiti 第 1 章\hspace{1em} 前言}
\cmdxmp{section[精简标题]}{这个题目实在太长了放到目录里面不太好看}{\heiti 1.1
  \hspace{1em} 这个题目实在太长了放到目录里面不太好看}
\cmdxmp{footnote}{我是可爱的脚注}{前方高能\footnote{我是可爱的脚注}}
\end{frame}

\begin{frame}[fragile]{\LaTeX{} 环境举例}
  \begin{minipage}{0.4\linewidth}
    \begin{lstlisting}[basicstyle=\ttfamily\small]
\begin{itemize}
  \item 一条
  \item 次条
  \item 这一条可以分为 ...
    \begin{itemize}
      \item 子一条
    \end{itemize}
\end{itemize}
\end{lstlisting}
  \end{minipage}\hspace{1.5cm}
  \begin{minipage}{0.4\linewidth}
\begin{itemize}
  \item 一条
  \item 次条
  \item 这一条可以分为 ...
    \begin{itemize}
      \item 子一条
    \end{itemize}
\end{itemize}
  \end{minipage}
\medskip

  \begin{minipage}{0.4\linewidth}
\begin{lstlisting}
\begin{enumerate}
  \item 一条
  \item 次条
  \item 再条
\end{enumerate}
\end{lstlisting}
  \end{minipage}\hspace{1.5cm}
  \begin{minipage}{0.4\linewidth}
\begin{enumerate}
  \item 一条
  \item 次条
  \item 再条
\end{enumerate}
  \end{minipage}
\end{frame}
%

\begin{frame}[fragile]{\LaTeX{} 数学公式}

\begin{columns}
\begin{column}{.5\textwidth}
\begin{lstlisting}[basicstyle=\ttfamily\small]
$V = \frac{4}{3}\pi r^3$

\[
  V = \frac{4}{3}\pi r^3
\]

\begin{equation}
\label{eq:vsphere}
V = \frac{4}{3}\pi r^3
\end{equation}
\end{lstlisting}
\end{column}

\begin{column}{.5\textwidth}
$V = \frac{4}{3}\pi r^3$

\[
  V = \frac{4}{3}\pi r^3
\]

\begin{equation}
\label{eq:vsphere}
V = \frac{4}{3}\pi r^3
\end{equation}
\end{column}
\end{columns}

\end{frame}

\begin{frame}[fragile]{\LaTeX{} 数学公式}
\begin{itemize}
\item 数学公式排版是 \LaTeX{} 的绝对强项
\item 数学排版需要进入数学模式,引用 \texttt{amsmath} 宏包
	\begin{itemize}
	\item 用单个美元符号(\verb|$|) 包围起来的内容是 {\bf 行内公式}
  \item 用两个美元符号(\verb|$$|) (不推荐)或 \verb|\[ \]| 包围起来的是 {\bf 单行公式} 或 {\bf 行间公式}
	\item 使用数学环境,例如 \texttt{equation} 环境内的公式会自动加上编号,
		\texttt{align} 环境用于多行公式(例如方程组、多个并列条件等)
  \end{itemize}
\item 寻找符号
    \begin{itemize}
      \item 运行 \texttt{texdoc symbols} 查看符号表
      \item S. Pakin. \emph{The Comprehensive \LaTeX{} Symbol List}
            \link{https://ctan.org/pkg/comprehensive}
      \item 手写识别(有趣但不全):Detexify \link{http://detexify.kirelabs.org}
    \end{itemize}
\item MathType 也可以使用和导出 \LaTeX{} 公式(不推荐)
\end{itemize}
\end{frame}

\begin{frame}[fragile,label={frame:unicode-math}]{unicode-math:现代的数学输入方式}
\LaTeX{} 的公式确实很强大,但是……符号有点难记?

\pkg{unicode-math} 宏包提供了几乎所见即所得的公式输入(\SJTUThesis 默认使用):

\begin{itemize}
  \item 可直接输入各类符号对应的 Unicode 字符(需要使用 UTF-8 编码):
  
  \begin{columns}[c]
    \begin{column}{0.45\textwidth}
      \begin{lstlisting}[basicstyle=\ttfamily]
\begin{equation*}
∫ Γ(x) dx = ±∞
\end{equation*}
      \end{lstlisting}
    \end{column}\hspace{1em}
    \begin{column}{0.45\textwidth}
      \begin{equation*}
        ∫ Γ(x) dx = ±∞
      \end{equation*}
      \end{column}
    \end{columns}
  \item 使用 |symbf| 等命令自动处理字母的粗体、斜体等变体,不必引入额外宏包。
\end{itemize}

\begin{columns}[c]
  \begin{column}{0.45\textwidth}
    \begin{lstlisting}[basicstyle=\ttfamily]
\begin{align*}
\symbf{\beta} &= \beta \symbf{I} \\
\symbf{a} &= a \symbf{I}
\end{align*}
\end{lstlisting}
\end{column}\hspace{1em}
  \begin{column}{0.45\textwidth}
    \begin{align*}
      \symbf{\beta} &= \beta \symbf{I} \\
      \symbf{a} &= a \symbf{I}
    \end{align*}
  \end{column}
\end{columns}

\end{frame}

\begin{frame}[fragile]{层次与目录生成}
\begin{columns}
\begin{column}{.6\textwidth}

\begin{lstlisting}[basicstyle=\ttfamily\small]
\tableofcontents % 这里是目录
\part{有监督学习}
\chapter{支持向量机}
\section{支持向量机简介}
\subsection{支持向量机的历史}
\subsubsection{支持向量机的诞生}
\paragraph{一些趣闻}
\subparagraph{第一个趣闻}
\end{lstlisting}
\end{column}
\begin{column}{.4\textwidth}
第一部分\quad 有监督学习\\
第一章\quad 支持向量机 \\
1. 支持向量机简介 \\
1.1 支持向量机的历史 \\
1.1.1 支持向量机的诞生 \\
一些趣闻  \\
第一个趣闻
\end{column}
\end{columns}
\end{frame}


\begin{frame}[fragile]{列表与枚举}
\begin{columns}
\begin{column}{.6\textwidth}
\begin{lstlisting}[basicstyle=\ttfamily\small]
\begin{enumerate}
\item \LaTeX{} 好处都有啥
  \begin{description}
    \item[好用] 体验好才是真的好
    \item[好看] 强迫症的福音
    \item[开源] 众人拾柴火焰高
  \end{description}
\item 还有呢?
  \begin{itemize}
    \item 好处 1
    \item 好处 2
  \end{itemize}
\end{enumerate}
\end{lstlisting}
\end{column}
\begin{column}{.4\textwidth}
{\small
\begin{enumerate}
\item \LaTeX{} 好处都有啥
  \begin{description}
    \item[好用] 体验好才是真的好
    \item[好看] 治疗强迫症
    \item[开源] 众人拾柴火焰高
  \end{description}
\item 还有呢?
  \begin{itemize}
    \item 好处 1
    \item 好处 2
  \end{itemize}
\end{enumerate}
}
\end{column}
\end{columns}

\end{frame}


\begin{frame}[fragile]{交叉引用与插入插图}
  \begin{itemize}
  \item 给对象命名:图片、表格、公式等\\
  |\label{name}|
\item 引用对象\\
  |\ref{name}|
  \end{itemize}
\bigskip

  \begin{minipage}{0.7\linewidth}
    \begin{lstlisting}
交大校徽请参见图~\ref{fig:badge}。
\begin{figure}[htbp]
  \centering
  \includegraphics[height=.2\textheight]%
  {libicon.pdf}
  \caption{交大校徽。}
  \label{fig:badge}
\end{figure}
\end{lstlisting}
  \end{minipage}\hfill
  \begin{minipage}{0.3\linewidth}\centering
    {\songti 交大校徽请参见图~1。}\\[1em]
 \resizebox{!}{0.2\textheight}{\sjtubadge[cprimary]}\\
 {\footnotesize\heiti 图~1. 交大校徽。}
  \end{minipage}
\end{frame}

\begin{frame}[fragile]{交叉引用与插入表格}
  \begin{columns}
  \column{.6\textwidth}
  \begin{lstlisting}
\begin{table}[htbp]
   \caption{编号与含义}
   \label{tab:number}
   \centering
   \begin{tabular}{cl}
     \toprule
     编号 & 含义 \\
     \midrule
     1    & 第一 \\
     2    & 第二 \\
     \bottomrule
   \end{tabular}
\end{table}
公式~(\ref{eq:vsphere}) 中编号与含义
请参见表~\ref{tab:number}。
\end{lstlisting}
\column{.4\textwidth}
\centering
{\small 表~1. 编号与含义}\\[2pt]
\begin{tabular}{cl}\toprule
编号 & 含义 \\\midrule
1 & 第一\\
2  & 第二\\\bottomrule
\end{tabular}\\[5pt]

\normalsize 公式~(\ref{eq:vsphere})编号与含义请参见表~1。
  \end{columns}
\end{frame}

\begin{frame}[fragile]{浮动体}
\begin{itemize}
\item 初学者最“捉摸不透”的特性之一 \link{https://liam.page/2017/03/11/floats-in-LaTeX-basic}
\item 图片和表格有时会很大,在插入的位置不一定放得下,因此需要浮动调整
\item 避免在文中使用「下图」「上图」的说法,而是使用图表的编号,例如 |图~\ref{fig:fig1}| 。
\item |\begin{figure}[<位置>] 图片 \end{figure}|
  \begin{itemize}
  \item 位置参数指定浮动体摆放的偏好
  \item |h| 当前位置(here), |t| 顶部(top), |b| 底部(bottom), |p| 单独成页(p)
  \item |!h| 表示忽略一些限制,|H| 表示强制\alert{(强烈不建议,除非你知道自己在做什么)}
  \end{itemize}
\item 温馨提示:图标题一般在下方,表标题一般在上方
\end{itemize}
\end{frame}

\begin{frame}[fragile]
  \frametitle{作图与插图}
  \begin{itemize}
    \item 外部插入

      \begin{itemize}
        \item Mathematica、MATLAB
        \item PowerPoint、Visio、Adobe Illustrator、Inkscape
        \item Python \pkg{Matplotlib} 库、\texttt{Plots.jl}、R、Plotly 等
        \item draw.io \link{https://draw.io/}、ProcessOn \link{https://www.processon.com/} 等在线绘图网站
      \end{itemize}

    \item \TeX{} 内联

      \begin{itemize}
        \item Asymptote
        \item \alert{\pkg{pgf}/\pkg{TikZ}、\pkg{pgfplots}}
      \end{itemize}

    \item 插图格式

      \begin{itemize}
        \item 矢量图:|.pdf|
        \item 位图:|.jpg| 或 |.png|
        \item \alert{不再推荐 \texttt{.eps}}
        \item 不(完全)支持 |.svg|、|.bmp|
      \end{itemize}

    \item 一些参考:\link{https://www.zhihu.com/question/21664179}
                    \link{https://tex.stackexchange.com/q/158668}
                    \link{https://tex.stackexchange.com/q/72930}
  \end{itemize}
\end{frame}

\begin{frame}[fragile]{表格绘制}
  \begin{itemize}
    \item 使用 \pkg{booktabs}、\pkg{longtables}、\pkg{multirow} 等宏包
    \item 手动绘制表格确实比较令人头疼,且较难维护
    \item 推荐使用在线工具绘制后导出代码:\LaTeX{} Table Generator \link{https://www.tablesgenerator.com/latex_tables}
  \end{itemize}
\end{frame}

\begin{frame}[fragile]{文献引用}
  \begin{itemize}
    \item 新时期我国信息技术产业的发展 \cite{devoftech}
    \item 他改变了中国 \cite{thelegendofjiang}
  \end{itemize}
\end{frame}

\begin{frame}[fragile]
  \frametitle{宏包推荐(\textbf{先读文档}后使用)}
  \setlength{\leftmarginii}{1.5em}
  \vspace{-1.5em}
  \begin{multicols}{3}
    \begin{itemize}
      \item 必备

        \begin{itemize}
          \item \pkg{amsmath}
          \item \pkg{graphicx}
          \item \pkg{hyperref}
        \end{itemize}

      \item 样式

        \begin{itemize}
          \item \pkg{caption}
          \item \pkg{enumitem}
          \item \pkg{fancyhdr}
          \item \pkg{footmisc}
          \item \pkg{geometry}
          \item \pkg{titlesec}
        \end{itemize}

      \item 数学

        \begin{itemize}
          \item \pkg{bm}
          \item \pkg{mathtools}
          \item \pkg{physics}
          \item \pkg{unicode-math}
        \end{itemize}

      \item 表格

        \begin{itemize}
          \item \pkg{array}
          \item \pkg{booktabs}
          \item \pkg{longtable}
          \item \pkg{tabularx}
        \end{itemize}

      \item 插图、绘图

        \begin{itemize}
          \item \pkg{float}
          \item \pkg{pdfpages}
          \item \pkg{standalone}
          \item \pkg{subfig}
          \item \pkg{pgf}/\pkg{tikz}
          \item \pkg{pgfplots}
        \end{itemize}

      \item 字体

        \begin{itemize}
          \item \pkg{newpx}
          \item \pkg{pifont}
          \item \pkg{fontspec}
        \end{itemize}

      \item 各种功能

        \begin{itemize}
          \item \pkg{algorithm2e}
          \item \pkg{beamer}
          \item \pkg{biblatex}
          \item \pkg{listings}
          \item \pkg{mhchem}
          \item \pkg{microtype}
          \item \pkg{minted}
          \item \pkg{natbib}
          \item \pkg{siunitx}
          \item \pkg{xcolor}
        \end{itemize}

      \item 多语言

        \begin{itemize}
          \item \pkg{babel}
          \item \pkg{polyglossia}
          \item \pkg{ctex}
          \item \pkg{xeCJK}
        \end{itemize}
    \end{itemize}
  \end{multicols}
  \vspace*{-0.5cm}
\end{frame}

\subsection{论文模板使用}

\begin{frame}{模板是什么?}
  \begin{itemize}
    \item 模板
      \begin{itemize}
        \item 已经设计好的格式框架
        \item 好的模板:使用户专注于内容
        \item 不应将时间花费在调整框架上
      \end{itemize}
    \item 再提 Office 和 Word
      \begin{itemize}
        \item 很少有人会有意识地在 Word 中使用模板
        \item 定义自己的标题?定义自己的列表?定义自己的段落样式?
        \item 自动化,还是手工调?
        \item 经常被折腾的精疲力竭
        \item 学习 \LaTeX{} 能帮助自己更好科学地使用 Word
      \end{itemize}
  \end{itemize}
\end{frame}

\begin{frame}{论文排版}
  \begin{itemize}
    \item 获取模板
      \begin{itemize}
        \item 随发行版自带、手动网络下载
        \item 模板文档类 \texttt{.cls} 文件
        \item 示例 \texttt{.tex} 文件
      \end{itemize}
    \item 编辑 \texttt{.tex} 文件:添加用户内容
    \item 编译:生成 PDF 文档
  \end{itemize}
\end{frame}

\begin{frame}[fragile]{论文排版举例}
  \begin{exampleblock}{IEEE 期刊论文}
    \begin{itemize}
      \item 获取模板:已随发行版自带
        \begin{itemize}
          \item 在安装目录 |<prefix>/texlive/2021/texmf-dist/doc/latex/IEEEtran|
          下找到 \verb"bare_jrnl.tex"
          \item 复制到某个文件夹(比如个人存论文的目录)
        \end{itemize}
      \item 编辑 \verb"bare_jrnl.tex" 文件 (英文模板:不支持中文)
      \item 编译
        \begin{itemize}
          \item 英文文献:\XeLaTeX{}、\pdfLaTeX{} 编译均可
        \end{itemize}
    \end{itemize}
  \end{exampleblock}
\end{frame}

\documentclass{ltxdoc}
\usepackage[scheme=plain]{ctex}
\usepackage[colorlinks]{hyperref}
\usepackage{iftex}
\ifPDFTeX % Windows User should always use this.
\else%
    \usepackage{fontspec}
    \defaultfontfeatures{Extension=.otf}
\fi
\usepackage{fontawesome}
\usepackage{tcolorbox}
\tcbuselibrary{skins}

\tcbset{
    enhanced,
    notitle,
    sharp corners,
    colframe=yellow,
    boxrule=1pt,
    center
}

\newtcolorbox{warn}[1][]{
    colback=yellow!5,
    borderline west={2mm}{0mm}{yellow},
    before upper={\textcolor{yellow}{\large\faWarning\ }},
    #1
}

\usepackage{listings}
\lstset{
    basicstyle=\ttfamily, 
    frame=single, 
    columns=flexible, 
    extendedchars=false
}

\def\fcmd#1{\paragraph{\fbox{\ttfamily #1}}}

\def\themename{\textsf{SJTUBeamer}}
\title{Development Guide of\\\themename}

\begin{document}
    \maketitle
    \tableofcontents
    \clearpage

    \section{Build}

    \subsection{l3build}

    \themename{} adopts \verb"l3build" system to build the package. After entering the source code directory:
    \begin{verbatim}
        cd src
    \end{verbatim}
    You could operate the package by the following command. The build script is set in \verb"build.lua".

    \fcmd{l3build doc} This command will generate the documentation of this package. We set an overall testing file \verb"doc/min.tex" to test all features of \themename. This is the most common command for testing a certain feature.
    \begin{center}
        unpack $\rightarrow$ \verb"min.tex" $\rightarrow$ documentations
    \end{center}

    \fcmd{l3build check} Process a regression test, with an optional parameter to perform one of the following tests:
    \begin{center}
        \ttfamily color font inner outer sjtuvi
    \end{center}
    which will show if the module is standing-free. If the compilation error occurs, please check if your code uses one definition from another module without requiring it first, or undefined control sequence occurs.

    \fcmd{l3build save color} If you make some modification to the framework (except \verb"sjtucover") and change the result. To save the current result as a reference, run this command and \verb"color" could be exchanged by any of the test units above.

    \fcmd{l3build clean} Clean the current \verb"src/build" directory, which will be useful if you modified one of the temporary files in that directory. (In this case, you may perform a wrong test with a modified version since \verb"l3build doc" may not refresh it.)

    \fcmd{l3build tag v1.5.0} Tag the version of the source code, which will also update the date in the source code.

    \fcmd{l3build ctan} This is the last step before release. Both \verb"l3build check" and \verb"l3build doc" will be performed. The generated file is \verb"src/sjtubeamer-ctan.zip".

    More information about \verb"l3build", please refer to
    \begin{verbatim}
        texdoc l3build
    \end{verbatim}

    \subsection{Customize Generation}\label{sec:precompile}

    In \verb"source/beamerthemesjtubeamer.ins", you could customize what templates you want to output on the line:
    \begin{verbatim}
        \def\preserveoption{,maxplus,max,min,my} 
    \end{verbatim}
    Reduce the parameter number will reduce the number of templates it generates. It is useful if you want to debug or make a slight performance jump. Or make your own version of this template.

    \begin{warn}
    \textbf{When you are pulling a request, the item in this column cannot be changed}. The extraction on your template should be decided by end users since large generation files could cause a performance drop on compiling. We are also glad to add your template to this column if your template is picked.
    \end{warn}

    \subsection{Pull Request}

    Before making a pull request, please refresh the root \verb".sty" files first in order to pass CI. In fact, if you are using \verb"l3build", we have embeded this action in \verb"l3build check" (which is included in \verb"l3build ctan"). But we still recommend that you should make a final run on the script. Switch to the main directory:
    \begin{verbatim}
        cd ..
    \end{verbatim}
    Then, run the bash code on *nix.
    \begin{verbatim}
        .github/ci/build_package.sh
    \end{verbatim}

    And all other scripts in the folder could also be checked in your local machine to make sure you can pass the CI on GitHub Actions.

    If you are a Windows user\footnote{Windows 10 users could also install WSL (Windows Subsystem for Linux) in order to use Linux environment conveniently. This also provides a faster speed on compiling.} and your \verb"l3build" distribution is not 2021/08/28 and later, please use the old extracting method in \verb"src/source":
    \begin{verbatim}
        latex beamerthemesjtubeamer.ins
    \end{verbatim}
    and copy the corresponding generated files to the root directory.

    \subsection{Check Integration}

    TBD.

    \section{Coding Style}

    This section gives a contribution code on coding style. Every contributor should follow this coding style if you want to make a pull request to this project.

    \subsection{Framework Code}

    The coding style in this section is available for the following files:
    \begin{verbatim}
        beamerthemesjtubeamer.dtx
        beamercolorthemesjtubeamer.dtx
        beamerfontthemesjtubeamer.dtx
        beamerinnerthemesjtubeamer.dtx
        beamerouterthemesjtubeamer.dtx
        sjtuvi.dtx
    \end{verbatim}

    \subsubsection{Support Version}

    We only support \TeX{} or \LaTeX{} code on these files. No \LaTeX3 code supported since \verb"beamer" is not written in \LaTeX3 and this template has a major dependency on \verb"beamer" class.

    \subsubsection{Kernel Function Definition}

    Kernel function means that it is not recommended to be used by end users.

    \paragraph{If your definition parameters are mandatory, use \TeX\ style.} This gives you better flexibility on the definition. And in this scenario, please use a \emph{unique} command name for your function.

    \begin{lstlisting}
\def\EqualOption#1#2#3{}
    \end{lstlisting}

    \subparagraph{Pros:} Using the bottom layer on \TeX{} gives better processing speed. And make the parameter declared obviously will make the code more readable, so as the end user.

    \subparagraph{Cons:} Using \TeX{} \verb"\def" method will skip the defined macro check provided by \LaTeX{}, which may cause macro conflict between different packages.

    \subparagraph{Decision:} Use complicated name if you want to avoid naming conficts on function. This template won't provide a mandatory rule on this.

    \paragraph{If your definition has an optional parameter, use \LaTeX\ style.} The maximum number of optional parameter is 1 only. You cannot have more optional parameters. Give the default optional parameter in the second square brackets.

    \begin{lstlisting}
\expandafter\providecommand\csname #1\endcsname
    [1][\sjtubeamer@logocolor]{}
    \end{lstlisting}

    \subparagraph{Pros:} Using this method could get an optional parameter.

    \subparagraph{Cons:} This method is less readable. And have to decide to use \verb"\newcommand" (check on duplicate functions) or \verb"\providecommand" (skip uniquness check).

    \subparagraph{Decision:} Just use this method if you want an optional parameter. You should not write a function with more than one optional parameters.

    \subsubsection{Interface Function Definition}

    Interface function means that the function will be used by end users.

    \paragraph{Use \LaTeX\ style on function definition.} This gives a clear visual indication on the type of this function.

    \begin{lstlisting}
\newcommand{\stamparray}[3]{}
    \end{lstlisting}

    \subparagraph{Pros:} This method will avoid the conflict on packages. Or it is easy to debug to know which package conflicts.

    \subparagraph{Cons:} This method will limit some features \TeX{} provides and you may not pass the compilation if you switch some \verb"\def" to \verb"\newcommand" or \verb"\providecommand".

    \subparagraph{Decision:} Use unique name on interface function.

    \subsubsection{Global Variable Definition}

    Global variable means that it will be used across modules.

    \paragraph{Global variables should use the naming scheme as follows.}

    \begin{itemize}
        \item Start with the project name \verb"sjtubeamer".
        \item Split by \verb"@" symbol and move to the next level. (only for middle layer)
        \item The final level should be the variable name itself.
    \end{itemize}

    \begin{lstlisting}
\def\sjtubeamer@logocolor{sjtuRedPrimary}
    \end{lstlisting}

    \subparagraph{Pros:} Make a clear visual indication of the variable type and avoid duplicates with other packages.

    \subparagraph{Cons:} A long name may make the typing experience troubling and easy to make faults (hard to check). 

    \subparagraph{Decision:} You should avoid using global variables, which will only be useful if you are making communication between modules. If a bug occurs, it is hard to track in \LaTeX{}. As a matter of fact, only the parameter in the function can be called a \emph{local variable} (with an immediate substitution). All variables defined by \verb"\def" will be stored in the global register.

    \subsubsection{Leaf Option Definition}

    Leaf option means it is an option that will not be passed to the next level, or could be used immediately in this module.

    \paragraph{Define the package option by a priority command, which will define a global variable as above.} This will create an option on this package, in package \#1, module \#2, and variable \#3.

    \begin{lstlisting}
\DefineOption{color}{color}{red}
    \end{lstlisting}
    which will define \verb"\def\sjtubeamer@color@color{red}" when it receives \verb"red" option. It is called a leaf option since it will be used in pair with the following condition test:

    \begin{lstlisting}
\if\EqualOption{color}{color}{red}\fi
    \end{lstlisting}
    which will check if the option is \verb"red".

    \subsubsection{Communication Option Definition}

    Communication option means it will be a global option passed between modules.

    \paragraph{Declare the variable in an obvious way.} Use \verb"beamer" macro to declare an option that will not be used in this module.

    \begin{lstlisting}
\DeclareOptionBeamer{red}{\def\sjtubeamer@color{red}}
    \end{lstlisting}

    \subparagraph{Pros:} We are making difference between leaf options and communication options in order to not mess up the communication model. It will make the code much simpler.

    \subparagraph{Cons:} It is not using \verb"kvoptions" package for a standard option receiving. Developer might to read the implementation of the macros in \verb"sjtuvi".

    \subparagraph{Decision:} Use our \emph{dialect} of option representation.

    \subsection{Cover Code}

    The coding style in this section is available for the following file:
    \begin{verbatim}
        sjtucover.dtx
    \end{verbatim}

    \subsubsection{Use Beamer Variable}

    \paragraph{Use beamer variable to set color and font.} \verb"beamer" class provides interfaces for using current configuration of color and font, see the documentation of \verb"beamer" class
    \begin{verbatim}
        texdoc beamer
    \end{verbatim}

    \begin{lstlisting}
\usebeamercolor{palette primary}
    \end{lstlisting}

    \subparagraph{Pros:} Use native macro will provide a much more flexibility of your cover template. It is essential to make your template fit in the color/font management system.

    \subparagraph{Cons:} Developer has to learn how to use these macros and some are not interested in that.

    \subparagraph{Decision:} To make a pull request, you should always avoid using fixed color/font.

    \subsubsection{New Option Declaration}

    \paragraph{Copy every region in every file that marked as \texttt{my} and paste just below it.} The region is marked as

    \begin{verbatim}
        <*my>
        ...
        </my>
    \end{verbatim}

    paste it and change \verb"my" to your name of your template. And add the precompiling option mentioned in Section \ref{sec:precompile}.

    \subparagraph{Pros:} This is useful for template management and end user could select what template to be precompiled.
    
    \subparagraph{Cons:} This is not so handy to search every file. We are working on an automated tool for developer to achive that.

    \subparagraph{Decision:} Follow the rule to submit your template.

    \subsubsection{Logo Color System}

    \paragraph{Change the variable to change the color logo.} This will change the \emph{temporary} variable of the logo color to fit with different scenarios.

    \begin{lstlisting}
\def\sjtubeamer@logocolor{palette primary.fg}
    \end{lstlisting}

    What's more, you should always use the placeholder to insert your logo, surrounded by a size constraint.
    \begin{lstlisting}
\resizebox{!}{1cm}{\insertlogo}
    \end{lstlisting}

    \begin{warn}
        Logo and title graphic share the color system on this variable. Be sure to change the color before you make those placeholders.
    \end{warn}

    \subsection{Comment}

    Comment means that it is in the Doc\TeX{} source code.

    \subsubsection{Macro Comment}

    \paragraph{Start a macro environment to start comment a macro.} This environment has one parameter to indicate the name of macro.

    \begin{lstlisting}
% \begin{macro}{\sjtubeamer@color}
    ...
% \end{macro}
    \end{lstlisting}

    \subsubsection{Macrocode}

    \paragraph{To insert source code, you should surround your code with macrocode environment.} This environment must start with four spaces after \verb"%".

    \begin{lstlisting}[showspaces=true]
%    \begin{macrocode}
...
%    \end{macrocode}\end{lstlisting}

    \section{Implementation}
% The DocTeX follows the alphabatic order to input.
    \DocInput{beamercolorthemesjtubeamer.dtx}
    \DocInput{beamerfontthemesjtubeamer.dtx}
    \DocInput{beamerinnerthemesjtubeamer.dtx}
    \DocInput{beamerouterthemesjtubeamer.dtx}
    \DocInput{beamerthemesjtubeamer.dtx}
% SJTUG doesn't hold the copyright of the following code.
    \DocInput{sjtucover.dtx}
    \DocInput{sjtuvi.dtx}
\end{document}
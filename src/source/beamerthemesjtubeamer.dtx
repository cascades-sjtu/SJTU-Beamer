% \iffalse meta-comment ---------------------------------------------
% Copyright (C) 2021 SJTUG
%
% Licensed under the Apache License, Version 2.0 (the "License");
% you may not use this file except in compliance with the License.
% You may obtain a copy of the License at
%
%     http://www.apache.org/licenses/LICENSE-2.0
%
% Unless required by applicable law or agreed to in writing, software 
% distributed under the License is distributed on an "AS IS" BASIS,
% WITHOUT WARRANTIES OR CONDITIONS OF ANY KIND, either express or implied.
% See the License for the specific language governing permissions and
% limitations under the License.
%
% ------------------------------------------------------------------- \fi
% \CheckSum{0}
% \StopEventually{}
% \iffalse
%
% Target Structure:
% +------------------------------------------------+
% +                    main.tex                    +
% +------------------------------------------------+
% +                   sjtubeamer                   +
% +------------------------------------------------+
% +    color    +   inner   +   outer  +    font   + 
% +             +-----------+          +           +
% +             +   cover   +          +           +
% +------------------------------------+-----------+
% +             sjtuvi                 +
% +------------------------------------+
% +              logo                  +
% +------------------------------------+
%
%<*package>
% ------------------------------------------------------------------- \fi
% \iffalse
\NeedsTeXFormat{LaTeX2e}
\ProvidesPackage{beamerthemesjtubeamer}[2021/08/21 sjtubeamer parent theme v1.3]
% \fi
%
% \subsection{Parent Theme}
%
% The primary job of this package is to load the component sub-packages of the
% \themename theme and route the theme options accordingly. It also
% provides some custom commands and environments for the user.
%
%   \begin{macro}{\sjtubeamer@cover}
%   This macro selects the cover theme.
%\begin{description}
%    \item[maxplus] The titlegraphic will be a photo.
%    \item[max] The background will be the photo.
%    \item[min] The design will be minimalistic.
%    \item[my]  Reserved interface for developers for customized title page and bottom page.
%\end{description}
%   
%   $\rightarrow$ The following will demostrates how to pass options between files.
%
%   $\rightarrow$ To set up an option, use \verb"\DeclareOptionBeamer" to let beamer theme receive such an option and store the value into a variable.
%
%   $\rightarrow$ To make the variable easy to follow and avoid duplicates, the naming system is as follows:
%   \begin{itemize}
%    \item Start with the project name \verb"sjtubeamer".
%    \item Split by \verb"@" symbol and move to the next level.
%    \item The final level should be the variable name itself.
%   \end{itemize}
%    \begin{macrocode}
\DeclareOptionBeamer{maxplus}{\def\sjtubeamer@cover{maxplus}}
\DeclareOptionBeamer{max}{\def\sjtubeamer@cover{max}}
\DeclareOptionBeamer{min}{\def\sjtubeamer@cover{min}}
\DeclareOptionBeamer{my}{\def\sjtubeamer@cover{my}} % reserved for customization
%    \end{macrocode}
%   $\rightarrow$ Next, declare the initial configuration.
%    \begin{macrocode}
\ExecuteOptionsBeamer{max}
%    \end{macrocode}
%
%  \begin{macro}{\sjtubeamer@color}
%   Choose the main color palette.
%   \begin{description}
%       \item[red] The red color palatte. \verb"sjtuRed*"
%       \item[blue] The blue color palatte. \verb"sjtuBlue*"
%   \end{description}
%
%    \begin{macrocode}
\DeclareOptionBeamer{red}{\def\sjtubeamer@color{red}}
\DeclareOptionBeamer{blue}{\def\sjtubeamer@color{blue}}
\ExecuteOptionsBeamer{red}
%    \end{macrocode}
%  \end{macro}
%
%  \begin{macro}{\sjtubeamer@lum}
%    Decide whether it is in light mode or dark mode. Switch the lumination.
%    \begin{macrocode}
\DeclareOptionBeamer{light}{\def\sjtubeamer@lum{light}}
\DeclareOptionBeamer{dark}{\def\sjtubeamer@lum{dark}}
\ExecuteOptionsBeamer{dark}
%    \end{macrocode}
%  \end{macro}
%
%  \begin{macro}{\sjtubeamer@lang}
%    Set the main language of this beamer.
%    \begin{macrocode}
\DeclareOptionBeamer{cn}{\def\sjtubeamer@lang{cn}}
\DeclareOptionBeamer{en}{\def\sjtubeamer@lang{en}}
\ExecuteOptionsBeamer{cn}
%    \end{macrocode}
%  \end{macro}
%
%  \begin{macro}{\sjtubeamer@nav}
%   Decide the type of the navigation bar.
%   \begin{description}
%       \item[miniframes] shows the progress on subsection.
%       \item[infolines] shows the page number and the document info.
%       \item[sidebar] make a sidebar to display the TOC. 
%   \end{description}
%    \begin{macrocode}
\DeclareOptionBeamer{miniframes}{\def\sjtubeamer@nav{miniframes}}
\DeclareOptionBeamer{infolines}{\def\sjtubeamer@nav{infolines}}
\DeclareOptionBeamer{sidebar}{\def\sjtubeamer@nav{sidebar}}
\ExecuteOptionsBeamer{miniframes}
%    \end{macrocode}
%  \end{macro}
%
%   $\rightarrow$ Then, the user's option will be processed after this (at the end of this subsubsection).
%    \begin{macrocode}
\ProcessOptionsBeamer
%    \end{macrocode}
%   $\rightarrow$ Finally, the option can only be one of above and pass it to the target package.
%    \begin{macrocode}
\PassOptionsToPackage{\sjtubeamer@cover}{beamerinnerthemesjtubeamer}
\PassOptionsToPackage{\sjtubeamer@color}{beamercolorthemesjtubeamer}
\PassOptionsToPackage{\sjtubeamer@lum}{beamercolorthemesjtubeamer}
\PassOptionsToPackage{\sjtubeamer@lang}{beamerouterthemesjtubeamer}
\PassOptionsToPackage{\sjtubeamer@nav}{beamerouterthemesjtubeamer}
%    \end{macrocode}
%
%   The following code is executed when it is in <presentation> mode.
%    \begin{macrocode}
\mode<presentation>
%    \end{macrocode}
%   Load sub-themes.
%    \begin{macrocode}
\usecolortheme{sjtubeamer}
\usefonttheme{sjtubeamer}
\useinnertheme{sjtubeamer}
\useoutertheme{sjtubeamer}
%    \end{macrocode}
%
% \iffalse
%</package>
% ------------------------------------------------------------------- \fi
% \Finale
\endinput

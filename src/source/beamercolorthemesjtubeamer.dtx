% \iffalse meta-comment --------------------------------------------------
% Copyright (C) 2021,2022 SJTUG
%
% Licensed under the Apache License, Version 2.0 (the "License");
% you may not use this file except in compliance with the License.
% You may obtain a copy of the License at
%
%     http://www.apache.org/licenses/LICENSE-2.0
%
% Unless required by applicable law or agreed to in writing, software 
% distributed under the License is distributed on an "AS IS" BASIS,
% WITHOUT WARRANTIES OR CONDITIONS OF ANY KIND, either express or implied.
% See the License for the specific language governing permissions and
% limitations under the License.
% ------------------------------------------------------------------------ \fi
% \iffalse
%<*package>
\NeedsTeXFormat{LaTeX2e}
\ProvidesPackage{beamercolorthemesjtubeamer}[2022/03/10 sjtubeamer color theme v2.5.6]
%</package>
% \fi
% \CheckSum{0}
% \StopEventually{}
% \iffalse
%<*package>
% ----------------------------------------------------------------------- \fi
%
% \subsection{Color Theme}
%
%   Load SJTU VI Library to get the pre-defined color.
%    \begin{macrocode}
\RequirePackage{sjtuvi}
%    \end{macrocode}
%
%  \begin{macro}{\sjtubeamer@color@color}
%    \begin{macrocode}
\DefineOption{color}{color}{red}
\DefineOption{color}{color}{blue}
\ExecuteOptionsBeamer{red}
%    \end{macrocode}
%  \end{macro}
%
%  \begin{macro}{\sjtubeamer@color@lum}
%    \begin{macrocode}
\DefineOption{color}{lum}{light}
\DefineOption{color}{lum}{dark}
\ExecuteOptionsBeamer{dark}
%    \end{macrocode}
%  \end{macro}
%
%    \begin{macrocode}
\ProcessOptionsBeamer
%    \end{macrocode}
%
%   Intermediate color palette depending on the choice of color theme.
%
%   NOTICE: inner theme has a replica of the following code.
%   To fullfill with the standfree requirements.
%    \begin{macrocode}
\if\EqualOption{color}{color}{red}
  \colorlet{cprimary}{sjtuRedPrimary}
  \colorlet{csecondary}{sjtuRedSecondary}
  \colorlet{ctertiary}{sjtuRedTertiary}
  \colorlet{cquanternary}{black}
\else
  \colorlet{cprimary}{sjtuBluePrimary}
  \colorlet{csecondary}{sjtuBlueSecondary}
  \colorlet{ctertiary}{sjtuBlueTertiary}
  \colorlet{cquanternary}{white}
\fi
%    \end{macrocode}
%
%   Main color palatte depending on the choice of lumination.
%   Especially, the \verb"structure" interface could not be derived from 
%   any other color palette.
%    \begin{macrocode}
\setbeamercolor{structure}{fg=cprimary}
\if\EqualOption{color}{lum}{light}
  \setbeamercolor{palette primary}{bg=white,fg=cprimary}
  \setbeamercolor{palette secondary}{bg=white,fg=cprimary!50!csecondary}
  \setbeamercolor{palette tertiary}{bg=white,fg=csecondary}
\else
  \setbeamercolor{palette primary}{bg=cprimary,fg=white}
  \setbeamercolor{palette secondary}{bg=cprimary!50!csecondary,fg=white}
  \setbeamercolor{palette tertiary}{bg=csecondary,fg=white}
\fi
%    \end{macrocode}
%
%   This part defines the color scheme of title.
%    \begin{macrocode}
\setbeamercolor{background canvas}{bg=white}
\setbeamercolor{normal text}{fg=black,bg=black!40}
\setbeamercolor*{block title}{parent=structure}
\setbeamercolor{titlelike}{bg=,fg=cprimary}
\setbeamercolor{title}{use={palette primary},fg=palette primary.fg,bg=}
\setbeamercolor{subtitle}{use={palette secondary},fg=palette secondary.fg,bg=}
\setbeamercolor{author}{use={palette primary},bg=,fg=palette primary.fg}
\setbeamercolor{institute}{parent=author}
\setbeamercolor{date}{parent=author}
%    \end{macrocode}
%
%   Logo color system is now controlled by \verb"logo.fg".
%   If you want to use a logo with its color definition in \verb"logo", use \verb"\insertlogo". Otherwise, if you want to use a logo that matches the current text color, use \verb"\usebeamertemplate{logo}".
%    \begin{macrocode}
\setbeamercolor{logo}{bg=,fg=cprimary}
%    \end{macrocode}
%
%   This part defines the color of block title.
%    \begin{macrocode}
\setbeamercolor{block title}{fg=white,bg=cprimary!90}
\setbeamercolor{block title alerted}{use=alerted text,
  fg=white,bg=csecondary}
\setbeamercolor{block title example}{use=example text,
  fg=cquanternary,bg=ctertiary}
%    \end{macrocode}
%
%   This part defines the color of block body.
%    \begin{macrocode}
\setbeamercolor{block body}{parent=normal text,use=block title,
  bg=block title.bg!30}
\setbeamercolor{block body alerted}{parent=normal text,
  use=block title alerted,bg=block title alerted.bg!30}
\setbeamercolor{block body example}{parent=normal text,
  use=block title example,bg=block title example.bg!30}
%    \end{macrocode}
%
%   This part defines the color of footline.
%    \begin{macrocode}
\setbeamercolor{footnote}{fg=cprimary,bg=}
%    \end{macrocode}
%
%   This part defines the color of part page, section page, and subsection page.
%    \begin{macrocode}
\setbeamercolor{part title}{parent={palette primary}}
\setbeamercolor{section title}{parent={palette secondary}}
\setbeamercolor{subsection title}{parent={palette tertiary}}
%    \end{macrocode}
%
%   Set the footline color.
%    \begin{macrocode}
\setbeamercolor{section in head/foot}{fg=white,bg=cprimary}
\setbeamercolor{subsection in head/foot}{fg=cprimary,bg=cprimary!20}
\setbeamercolor{institute in head/foot}{fg=white,bg=}
\setbeamercolor{page number in head/foot}{fg=white,bg=}
%    \end{macrocode}
%
%   Color patch for sidebar, shadow, and smoothtrees.
%   WARNING: This is a replica from outer theme, just in case if you want to use color theme seperately. But it is useless in the overall theming.
%    \begin{macrocode}
\setbeamercolor{frametitle}{use=titlelike,bg=white,fg=titlelike.fg}
\setbeamercolor{frametitle right}{parent=subsection in head/foot}
%    \end{macrocode}
%
%   Set the emphasized color and redefine the emphasizing command to make the text both italic for ASCII character and colored in the middle color of cprimary and csecondary.
%
%   The redefinition is required since beamer class has redefined the \verb"\emph" command to make it not nested. According to LearnLaTeX.org, the emphasized color is defined to make contrast in presentation.
%
%   For ASCII character, the italic part dominates, as it is quite different from the normal roman font. As for chinese character, the color part dominates, since it is often in bolder shape and changing to other font will make the layout messy.
%    \begin{macrocode}
\setbeamercolor{emph}{fg=cprimary!50!csecondary}
\renewcommand<>{\emph}[1]{%
  {\only#2{\usebeamercolor[fg]{emph}\itshape}#1}%
}
%    \end{macrocode}
%
%   As is native to beamer, you could also use \verb"\alert" command to highlight the text. The color is redirected to the cprimary.
%    \begin{macrocode}
\setbeamercolor{alerted text}{fg=cprimary}
%    \end{macrocode}
%
%
% \iffalse
%</package>
% \fi
% \Finale
\endinput

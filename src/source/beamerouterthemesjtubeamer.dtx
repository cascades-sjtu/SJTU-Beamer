% \iffalse meta-comment --------------------------------------------------
% Copyright (C) 2021,2022 SJTUG
%
% Licensed under the Apache License, Version 2.0 (the "License");
% you may not use this file except in compliance with the License.
% You may obtain a copy of the License at
%
%     http://www.apache.org/licenses/LICENSE-2.0
%
% Unless required by applicable law or agreed to in writing, software 
% distributed under the License is distributed on an "AS IS" BASIS,
% WITHOUT WARRANTIES OR CONDITIONS OF ANY KIND, either express or implied.
% See the License for the specific language governing permissions and
% limitations under the License.
% ------------------------------------------------------------------------ \fi
% \iffalse
%<*package>
\NeedsTeXFormat{LaTeX2e}
\ProvidesPackage{beamerouterthemesjtubeamer}[2022/03/09 sjtubeamer outer theme v2.5.5]
%</package>
% \fi
% \CheckSum{0}
% \StopEventually{}
% \iffalse
%<*package>
% ------------------------------------------------------------------- \fi
%
% \subsection{Outer Theme}
%
% A |beamer| outer theme dictates the style of the frame elements traditionally
% set outside the body of each slide: the head, footline, and frame title.
%
%   Load SJTU VI Library to get the definition on color and shape.
%    \begin{macrocode}
\RequirePackage{sjtuvi}
%    \end{macrocode}
%
%
%  \begin{macro}{\sjtubeamer@outer@nav}
%    \begin{macrocode}
\DefineOption{outer}{nav}{miniframes}
\DefineOption{outer}{nav}{infolines}
\DefineOption{outer}{nav}{sidebar}
\DefineOption{outer}{nav}{default}
\DefineOption{outer}{nav}{smoothbars}
\DefineOption{outer}{nav}{split}
\DefineOption{outer}{nav}{shadow}
\DefineOption{outer}{nav}{tree}
\DefineOption{outer}{nav}{smoothtree}
\ExecuteOptionsBeamer{miniframes}
%    \end{macrocode}
%  \end{macro}
%
% \begin{macro}{\sjtubeamer@outer@logopos}
%    \begin{macrocode}
\DefineOption{outer}{logopos}{topright}
\DefineOption{outer}{logopos}{bottomright}
\ExecuteOptionsBeamer{bottomright}
%    \end{macrocode}
% \end{macro}
%
%    \begin{macrocode}
\ProcessOptionsBeamer
%    \end{macrocode}
%
% Enable compress option on beamer to avoid multiline navigation dots.
%    \begin{macrocode}
\beamer@compresstrue
%    \end{macrocode}
%
% \subsubsection{Frame Title}\label{sec:frametitle}
%
%
%   Define the beamer template \verb"frametitle" to add a logo in the right of the frametitle (top right corner) for \verb"topright" option. If it is not \verb"topright" option, then the default \verb"frametitle" template is used.
%
%   \verb"smoothbars", \verb"smoothtree" and \verb"shadow" theme use a different frame title insertion mechanism in \verb"beamer" class so that the redefinition on \verb"frametitle" should be done just after the document starts. This overwrites the original definition of those outer theme and may cause a certain degree of style lost.
%    \begin{macrocode}
\if\EqualOption{outer}{logopos}{topright}
  \AtBeginDocument{
    \defbeamertemplate*{frametitle}{sjtubeamer}[1][left]
    {%
    \ifbeamercolorempty[bg]{frametitle}{}{\nointerlineskip}%
      \@tempdima=\textwidth%
      \advance\@tempdima by\beamer@leftmargin%
      \advance\@tempdima by\beamer@rightmargin%
      \begin{beamercolorbox}[sep=0.3cm,#1,wd=\the\@tempdima]{frametitle}
        \begingroup
        \usebeamerfont{frametitle}
        \vbox{}
        \ifx\insertframesubtitle\@empty\vskip-2pt%
        \else\vskip-1ex\fi%
        \if@tempswa\else\csname beamer@fte#1\endcsname\fi%
        \strut\insertframetitle\strut\par%
        {%
          \ifx\insertframesubtitle\@empty%
          \else%
          {
            \usebeamerfont{framesubtitle}
            \usebeamercolor[fg]{framesubtitle}
            \strut\insertframesubtitle\strut\par
          }%
          \fi
        }%
        \vskip-1ex%
        \endgroup%
        \raggedleft%
        \begingroup
        \ifx\insertframesubtitle\@empty\vskip-2.5ex%
        \else\vskip-3.5ex\fi%
%    \end{macrocode}
% Insert the outer logo with 0.7cm height. \verb"\vphantom" here is to create a phantom box to make sure the \verb"\resizebox" has a non-zero height box input. The calculation is fixed for a 0.7cm height logo placeholder.
%    \begin{macrocode}
        \resizebox{!}{0.7cm}{\vphantom{-}\usebeamertemplate{logo}}\hspace*{5pt}%
        \endgroup%
        \ifx\insertframesubtitle\@empty%
        \else\vskip0.5ex\fi%
        \if@tempswa\else\vskip-.3cm\fi%
      \end{beamercolorbox}%
    }
  }
\fi
%    \end{macrocode}
%
%  \subsubsection{Execute Outer Theme}
%
% Use the built-in outer templates. If you want to take special care about one outer theme, please add a condition test on that theme. Otherwise, it will use the default configuration of the corresponding built-in outer theme.
%    \begin{macrocode}
\if\EqualOption{outer}{nav}{miniframes}
  \useoutertheme[footline=institutetitle]{miniframes}
%    \end{macrocode}
% Sidebar.
%    \begin{macrocode}
\else\if\EqualOption{outer}{nav}{sidebar}
    \useoutertheme{sidebar}
%    \end{macrocode}
% Since the logo area in the sidebar theme is square, the default configuration for \verb"\logo" will be \verb"\sjtubadge" no matter what kind of cover processed in the inner theme. If you want to modify the logo later, change the logo manually after loading the theme.
%    \begin{macrocode}
    \logo{\resizebox{!}{1cm}{\sjtubadge}}
%    \end{macrocode}
% And it will use the logo position configuration \verb"topright" for sidebar to remove the logo in the bottom right corner. The config will not be modified after the outer theme is loaded.
%    \begin{macrocode}
    \def\sjtubeamer@outer@logopos{topright}
%    \end{macrocode}
% Use the default template for \verb"frametitle" in sidebar, since there has already been a logo in the top left corner. The setup has to be done after document starts since the previous set of \verb"frametitle" template is also set up after document as is shown in Section \ref{sec:frametitle}. Appending on this hook will override the previous setup.
%    \begin{macrocode}
    \AtBeginDocument{
      \setbeamertemplate{frametitle}[sidebar theme]
    }
%    \end{macrocode}
% If it is the other theme, change the beamercolor to fit smooth* theme.
%    \begin{macrocode}
  \else
    \useoutertheme{\sjtubeamer@outer@nav}
    \setbeamercolor{title in head/foot}{use=structure,bg=white,fg=structure.fg}
  \fi\fi
%    \end{macrocode}
%
%   Color patch for default outer theme.
%    \begin{macrocode}
\if\EqualOption{outer}{nav}{default}
  \setbeamercolor{frametitle}{use=palette primary,
    bg=palette primary.bg,fg=palette primary.fg}
%    \end{macrocode}
% Since \verb"infolines" outer theme in \verb"beamer" class has redefined the beamer color below which doesn't confirm the standing-free principle, the redefined color will override after the outer theme is loaded. 
% \verb"structure" beamer color stores the \verb"cprimary". Set the color relative to this for decoupling reasons.
%    \begin{macrocode}
\else\if\EqualOption{outer}{nav}{infolines}
  \setbeamercolor{author in head/foot}{use=structure,fg=white,bg=structure}
  \setbeamercolor{title in head/foot}{use=structure,fg=structure,bg=structure!10}
  \setbeamercolor{date in head/foot}{use=structure,fg=structure,bg=structure!20}
  \setbeamercolor{section in head/foot}{parent=author in head/foot}
  \setbeamercolor{subsection in head/foot}{parent=date in head/foot}
%    \end{macrocode}
% And \verb"infolines" also resize the text margin, we reset the margin to the default size in order to avoid unwanted shift in covers.
%    \begin{macrocode}
  \setbeamersize{text margin left=1cm,text margin right=1cm}
%    \end{macrocode}
%   Color patch for sidebar, shadow, and smoothtrees.
%   WARNING: This cannot be moved to the color theme, since the color was set in the outer theme itself. And the outer theme is the last theme to be loaded.
%    \begin{macrocode}
\else
  \setbeamercolor{frametitle}{use=titlelike,bg=white,fg=titlelike.fg}
  \setbeamercolor{frametitle right}{parent=subsection in head/foot}
\fi\fi
%    \end{macrocode}
%
% \subsubsection{Bottombar}
%
%   Clear the original definition of sidebar first. Then append the page info to the footline, which could avoid collision on footnote.
%    \begin{macrocode} 
\setbeamertemplate{sidebar right}{}
%    \end{macrocode}
%
% Add page number to the toolbox.
%    \begin{macrocode}
\addtobeamertemplate{navigation symbols}{}{
  \hbox{
    \raisebox{1.2pt}[0pt][0pt]{
      \usebeamerfont{footline}
      \usebeamercolor{footline}
      \color{footline.fg!50}
      \insertframenumber/\inserttotalframenumber
      \hspace*{0.2em}
    }
  }
}
%    \end{macrocode}
%
%  Append the navigation to the footline to avoid the collision with the navigation symbols. This will clear the original placeholder for the logo -- place it elsewhere. However, in the \verb"bottomright" mode, the logo size is not constrainted.
%  The code is a bit redundent since we don't want it to check the condition every time making a bottom bar.
%    \begin{macrocode}
\if\EqualOption{outer}{logopos}{bottomright}
  \addtobeamertemplate{footline}{
    \hfill%
    \usebeamertemplate***{navigation symbols}%
    \llap{\raisebox{1pc}[0pt][0pt]{\insertlogo}}
    \hspace*{0.1cm}\par
    \vskip 4pt
  }{}
\else
  \addtobeamertemplate{footline}{
    \hfill%
    \usebeamertemplate***{navigation symbols}%
    \hspace*{0.1cm}\par
    \vskip 4pt
  }{}
\fi
%    \end{macrocode}
%
% And set the page number template to none.
%    \begin{macrocode}
\setbeamertemplate{page number in head/foot}{}
%    \end{macrocode}
%
%
% \iffalse
%</package>
% \fi
% \Finale
\endinput

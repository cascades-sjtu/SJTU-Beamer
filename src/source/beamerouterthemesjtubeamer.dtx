% \iffalse meta-comment --------------------------------------------------
% Copyright (C) 2021 SJTUG
%
% Licensed under the Apache License, Version 2.0 (the "License");
% you may not use this file except in compliance with the License.
% You may obtain a copy of the License at
%
%     http://www.apache.org/licenses/LICENSE-2.0
%
% Unless required by applicable law or agreed to in writing, software 
% distributed under the License is distributed on an "AS IS" BASIS,
% WITHOUT WARRANTIES OR CONDITIONS OF ANY KIND, either express or implied.
% See the License for the specific language governing permissions and
% limitations under the License.
% ------------------------------------------------------------------------ \fi
% \iffalse
%<*package>
\NeedsTeXFormat{LaTeX2e}
\ProvidesPackage{beamerouterthemesjtubeamer}[2021/09/03 sjtubeamer outer theme v2.0.0]
%</package>
% \fi
% \CheckSum{0}
% \StopEventually{}
% \iffalse
%<*package>
% ------------------------------------------------------------------- \fi
%
% \subsection{Outer Theme}
%
% A |beamer| outer theme dictates the style of the frame elements traditionally
% set outside the body of each slide: the head, footline, and frame title.
%
%   Load SJTU VI Library to get the definition on color and shape.
%    \begin{macrocode}
\RequirePackage{sjtuvi}
%    \end{macrocode}
%
%
%  \begin{macro}{\sjtubeamer@outer@nav}
%    \begin{macrocode}
\DefineOption{outer}{nav}{miniframes}
\DefineOption{outer}{nav}{infolines}
\DefineOption{outer}{nav}{sidebar}
\DefineOption{outer}{nav}{default}
\DefineOption{outer}{nav}{smoothbars}
\DefineOption{outer}{nav}{split}
\DefineOption{outer}{nav}{shadow}
\DefineOption{outer}{nav}{tree}
\DefineOption{outer}{nav}{smoothtree}
\ExecuteOptionsBeamer{miniframes}
%    \end{macrocode}
%  \end{macro}
%
%    \begin{macrocode}
\ProcessOptionsBeamer
%    \end{macrocode}
%
% Enable compress option on beamer to avoid multiline navigation dots.
%    \begin{macrocode}
\beamer@compresstrue
%    \end{macrocode}
%
% Use the built-in outer templates. If you want to take special care about one outer theme, please add a condition test on that theme. Otherwise, it will use the default configuration of the corresponding built-in outer theme.
%    \begin{macrocode}
\if\EqualOption{outer}{nav}{miniframes}
  \useoutertheme[footline=institutetitle]{miniframes}
%    \end{macrocode}
% Sidebar.
%    \begin{macrocode}
\else\if\EqualOption{outer}{nav}{sidebar}
  \useoutertheme{sidebar}
%    \end{macrocode}
% If it is other theme, change the beamercolor to fit smooth* theme.
%    \begin{macrocode}
\else
  \useoutertheme{\sjtubeamer@outer@nav}
  \setbeamercolor{title in head/foot}{use=structure,bg=white,fg=structure.fg}
\fi\fi
%    \end{macrocode}
%
%   Color patch for sidebar, shadow, and smoothtrees.
%   WARNING: This cannot be moved to the color theme, since the color was set in the outer theme itself. And the outer theme is the last theme to be loaded.
%    \begin{macrocode}
\setbeamercolor{frametitle}{use=titlelike,bg=white,fg=titlelike.fg}
\setbeamercolor{frametitle right}{parent=subsection in head/foot}
%    \end{macrocode}
%
% \subsubsection{Bottombar}
%
%   Clear the original definition of sidebar first. Then append the page info to the footline, which could avoid collision on footnote.
%    \begin{macrocode} 
\setbeamertemplate{sidebar right}{}
%    \end{macrocode}
%
% Add page number to the toolbox.
%    \begin{macrocode}
\addtobeamertemplate{navigation symbols}{}{
  \hbox{
    \raisebox{1.2pt}[0pt][0pt]{
      \usebeamerfont{footline}
      \usebeamercolor{footline}
      \color{footline.fg!50}
      \insertframenumber/\inserttotalframenumber
      \hspace*{0.2em}
    }
  }
}
%    \end{macrocode}
%
%  Append the navigation to the footline to avoid the collision with the navigation symbols.
%    \begin{macrocode}
\addtobeamertemplate{footline}{
  \hfill%
  \usebeamertemplate***{navigation symbols}%
  \hspace*{0.1cm}\par
  \vskip 4pt
}{}
%    \end{macrocode}
%
% And set the page number template to none.
%    \begin{macrocode}
\setbeamertemplate{page number in head/foot}{}
%    \end{macrocode}
%
%
%
% \iffalse
% </package>
% \fi
% \Finale
\endinput
